%!TEX root = SISC_elastic_3d.tex
\section{Introduction}
Seismic wave propagation has important applications in earthquake simulation, energy resources exploration, and underground motion analysis. In many practical problems, wave motion is governed by the three dimensional (3D) anisotropic elastic wave equations. The layered structure of the Earth gives rise to a piecewise smooth material property with discontinuities at internal interfaces, which are often curved in realistic models. Because of the heterogeneous material property and internal interfaces, the governing equations cannot be solved analytically, and it is necessary to use advanced numerical techniques to solve the seismic wave propagation problem.

When solving hyperbolic partial differential equations (PDEs), for computational efficiency, it is essential that the numerical methods are high order accurate (higher than second order). This is because high order methods have much smaller dispersion error than lower order methods \cite{Hagstrom2012, Kreiss1972}. However, it is challenging to obtain a stable and high order accurate method in the presence of discontinuous material property and non-trivial geometry. 

Traditionally, the governing equations of seismic wave propagation are solved as a first order system, either in velocity-strain or velocity-stress formulation, which consists of nine equations. With the finite difference method, staggered grids are often used for first order systems, and recently the technique has been generalized to staggered curvilinear grids for the wave equation \cite{OReilly2020}. The finite difference method on non-staggered grids has also been developed for seismic wave simulation in 2D \cite{Kozdon2013} and 3D \cite{Duru2016}.

In this paper, we use another approach that discretizes the governing equations in second order form. Comparing with nine PDEs in a first order system, the second order formulation consists of only three PDEs in the displacement variables. In many cases, this could be a more efficient approach in terms of accuracy and memory usage. For spatial discretization, we consider the finite difference operators constructed in \cite{sjogreen2012fourth} that satisfy a summation-by-parts (SBP) principle, which is a discrete analog of the integration-by-parts principle and is an important ingredient to obtain energy stability. The SBP operators in \cite{sjogreen2012fourth} use a ghost point outside each boundary to impose boundary conditions strongly. The ghost point values are obtained by solving a system of linear equations. This can be avoided by imposing boundary conditions in a weak sense \cite{Carpenter1994} with the SBP operators constructed in \cite{Mattsson2012} that do not use any ghost point. The close relationship between these two types of SBP operators is explored in \cite{wang2018fourth}, where it was also shown in test problems that the approach using ghost points has better CFL property.

In the SBP finite difference framework, a multi-block approach is often taken when the material property is discontinuous. That is, the domain is divided into subdomains such that the internal interfaces are aligned with the material discontinuities. Each subdomain has four sides in 2D and six faces in 3D, which can then be mapped to a reference domain, for example, a unit square in 2D and a unit cube in 3D. In each subdomain, material properties are smooth and SBP operators are used independently for the spatial discretization of the governing equations. To patch subdomains together, physical interface conditions are imposed at internal interfaces \cite{Almquist2019,duru2014stable}. It is challenging to derive energy stable interface coupling with high order accuracy. 

In \cite{petersson2015wave}, a fourth order SBP finite difference method was developed to solve the 3D elastic wave equation in heterogeneous smooth media, where topography in non-rectangular domains is resolved by using curvilinear meshes. The main objective of the present paper is to develop a fourth order method that solves the governing equations in piecewise smooth media, where material discontinuities occur at curved interfaces.   This is motivated by the fact that in realistic models, material properties are only piecewise smooth with discontinuities, and it is important to obtain high order accuracy even at the material interfaces. A highlight of our method is that mesh sizes in each subdomain can be chosen according to the velocity structure of the material property. This leads to difficulties in mesh refinement interfaces, but maximizes computational efficiency. In the context of seismic wave propagation, as going deeper in the Earth, the wave speed gets larger and the wavelength gets longer. Correspondingly, in our model, the mesh becomes coarser with increasing depth. In this way, the number of grid points per wavelength can be kept almost the same in the entire domain. In addition, curved interfaces are also useful when the top surface has a very complicated geometry. If only planar interfaces are used \cite{SW4}, the size of the finest mesh block on top must be large to keep small skewness of the grid. With curved interfaces, the size of the finest mesh block can be reduced without increasing the skewness of the grid. 

In \cite{wang2018fourth}, we developed a fourth order finite difference method for the 2D wave equations with mesh refinement interfaces on Cartesian grids. Our current work generalizes to 3D elastic wave equations on curvilinear grids. In a 3D domain, the material interfaces are 2D curved faces. To impose interface conditions on hanging nodes, we construct fourth order interpolation and restriction operators for 2D grid functions. These operators are compatible with the underlying finite difference operators. With a fourth order predictor-corrector time integrator, the fully discrete discretization is energy conserving. 

The rest of the paper is organized as follows. In Sec.~2, we introduce the governing equations in curvilinear coordinates. The spatial discretization is presented in detail in Sec.~3. Particular emphasis is placed on the numerical coupling procedure at curved mesh refinement interfaces. In Sec.~4, we describe the temporal discretization and present the fully discrete scheme. Numerical experiments are presented in Sec.~5 to verify the convergence rate of the proposed scheme and the energy conserving property. We also demonstrate that the mesh refinement interfaces do not introduce spurious wave reflections. Conclusions are drawn in Sec.~6. 
