%\documentclass[12pt,reqno]{amsart}
\documentclass[twoside,11pt]{article}
\usepackage{amsmath,amssymb,amsthm,amsfonts}
\topmargin=-0.5in \oddsidemargin3mm \evensidemargin3mm
\textheight225mm \textwidth160mm
\usepackage{xcolor}
 \usepackage{natbib}
\usepackage{graphicx}
\linespread{1.125}
\begin{document}

\begin{center}
{\Large \textbf{Responses to Referee Report (Reviewer \#1)}}
\end{center}

\noindent{Dear editor and referees of the SIAM Journal on Scientific Computing},

We would like to thank you for the time and help in handling and carefully reading our submission and providing encouraging and constructive comments.  We have revised our paper accordingly and improved its context in several courses. 

We have added a numerical example of the benchmark problem LOH.1 in Section 5.5.

We have also carefully proofread the manuscript, polished its writing and corrected the typos and grammatical errors.  Here are the comments from reviewer \#1 in bold followed by our responses.

\begin{enumerate}
\item \textbf{How do you deal with an interface of really any shape, for example a spherical cavity?}\\
 We have added in the introduction section how interfaces can be constructed in the general setting. We divide the computational domain into subdomains such that each subdomain has four sides in 2D and six faces in 3D. We then map each curved subdomain into a reference domain.

\item  \textbf{At the interface between two different grids (but with identical physical properties), are parasitic reflections zero? It is likely that parasitic modes are generated; see for example the work of Collino-Fouquet-Joly ("A conservative space-time mesh refinement method for the 1-D wave equation") or Rodriguez-Joly. The maps in figure 6 are not convincing: cuts with and without interface would be much better suited}\\
 For Figure 6, the reference solutions are actually obtained from a uniform Cartesian grid without using any interface. This information has been added to the caption of Figure 6 and 7. From the pictures, we observe small reflections when mesh grids are coarse. But when we refine the mesh, we observe that the reflections become smaller. This is also evidenced in the LOH.1 test problem we have added: there is small difference between the numerical solution and the exact solution with a coarse mesh in Figure 9, but they look identical when we refine the mesh, see Figure 10.%Because the method converges with fourth order, the amplitude of reflections decreases fast with mesh refinement. %Ideally, we will not have reflection generated from curved interfaces when mesh size is samll enough. 

\end{enumerate}

Again, we appreciate you all for the time in handling and scrutinizing our manuscript, and thanks for considering our submission to this journal.

Sincerely, 

All authors



\end{document}
