%!TEX root = SISC_elastic_3d.tex
\section{The temporal discretization}
The equations are advanced in time with an explicit fourth order accurate predictor-corrector time integration method. Like all explicit time stepping methods, the time step must not exceed the CFL stability limit. By a similar analysis as in \cite{sjogreen2012fourth}, we require 
\begin{equation*}
\Delta_t\leq C_{\text{cfl}}\min\{h_1,h_2,h_3\}/\sqrt{\kappa_{\max}},
\end{equation*}
where $C_{\text{cfl}} = 1.3$ and
$\kappa_{\text{max}}$ is the maximum eigenvalue of the matrices 
\[T_{\bf i}^{\{f,c\}} = \frac{1}{\rho^{\{f,c\}}({\bf r}_{\bf i})}\left(\begin{array}{ccc}
Tr(N_{11}^{\{f,c\}}({\bf r}_{\bf i})) &  Tr(N_{12}^{\{f,c\}}({\bf r}_{\bf i}))& Tr(N_{13}^{\{f,c\}}({\bf r}_{\bf i}))\\
Tr(N_{21}^{\{f,c\}}({\bf r}_{\bf i})) & Tr(N_{22}^{\{f,c\}}({\bf r}_{\bf i})) & Tr(N_{23}^{\{f,c\}}({\bf r}_{\bf i}))\\
Tr(N_{31}^{\{f,c\}}({\bf r}_{\bf i})) & Tr(N_{32}^{\{f,c\}}({\bf r}_{\bf i})) & Tr(N_{33}^{\{f,c\}}({\bf r}_{\bf i}))\end{array}\right), \]
where $Tr(N_{lm}^{\{f,c\}}({\bf r}_{\bf i}))$ represents the trace of $3\times3$ matrix $N_{lm}^{\{f,c\}}({\bf r}_{\bf i})$. Note that $\kappa_{\text{max}}$ is related to the material properties $\mu^{\{f,c\}}, \lambda^{\{f,c\}}$ and $\rho^{\{f,c\}}$. The notation $\{\cdot,\cdot\}$ represents the component-wise identities. In the following, we give detailed procedures about how we apply the fourth order time integrator to the semidiscretizations (\ref{elastic_semi_c}) and  (\ref{fine_scheme}). 

Let ${\bf c}^{n}$ and ${\bf f}^{n}$ denote the numerical approximations of ${\bf c}({\bf x},t_n), {\bf x}\in\Omega^c$ and ${\bf f}({\bf x},t_n), {\bf x}\in\Omega^f$, respectively. Here, $t_n = n\Delta_t, n = 0,1,\cdots$ and $\Delta_t > 0$ is a constant time step. We present the fourth order time integrator with predictor and corrector in  Algorithm \ref{first_alg}.
~\\
\begin{breakablealgorithm}
	\caption{Fourth order accurate time stepping for the semidiscretizations  (\ref{elastic_semi_c}) and  (\ref{fine_scheme}). }\label{first_alg}
	Given  $\wt{{\bf c}}^{n}, \wt{{\bf c}}^{n-1}$ and ${\bf f}^{n}, {\bf f}^{n-1}$ that satisfy the discretized interface conditions.
	
	\begin{itemize}
		\item  {Compute the predictor at the interior grid points %for both fine and coarse domains
			\[{\bf c}^{*,n+1}_{\bf i} = 2{\bf c}^{n}_{\bf i} - {\bf c}^{n-1}_{\bf i} + \Delta_t^2\left((\rho^{2h}\otimes{\bf I})(J^{2h}\otimes{\bf I})\right)^{-1}{\wt{\mathcal{L}}}^{2h}_{\bf i} {\bf{c}}^{n},\quad {\bf i}\in I_{\Omega^c},\]
			\[{\bf f}^{*,n+1}_{\bf i} = 2{\bf f}^{n}_{\bf i} - {\bf f}^{n-1}_{\bf i} + \Delta_t^2\left((\rho^{h}\otimes{\bf I})(J^{h}\otimes{\bf I})\right)^{-1}\hat{\mathcal{L}}^{h}_{\bf i} {\bf{f}}^{n},\quad {\bf i}\in I_{\Omega^f}\backslash I_{\Gamma^f}.\]
		}
		\item{At the interface $\Gamma$, the values ${\bf f}^{*,n+1}_{\bf i}$  are computed by the continuity of solution 
			\begin{equation*}
			{\bf f}^{*,n+1}_{\bf i} = {\mathcal{P}}_{\bf i}({\bf c}^{*,n+1}),\quad {\bf i}\in I_{\Gamma^f}.
			\end{equation*}
		}
		\item{At the interface $\Gamma$, the ghost point values in $\wt{\bf c}^{*,n+1}$ are computed by solving the equation for the continuity of traction 
			%\begin{equation}\label{traction_gamma_pre}
			%({\Lambda}^{c}_{{\bf i}'}{J}_{{\bf i}'}^{c})^{-1}\wt{\mathcal{A}}_3^{2h}{\bf c}^{*,n+1}_{{\bf i}'}
			%= {\mathcal{R}}\left(({\Lambda}^{f}_{\bf i}{J}^f_{\bf i})^{-1}(\mathcal{A}_3^h{\bf f}^{*,n+1}_{\bf i}-h_3\omega_1{\bm \eta}^{\ast,n+1}_{\bf i})\right),\quad {\bf i}'\in I_{\Gamma^c},\quad {\bf i}\in I_{\Gamma^f}.
			%\end{equation}
				\begin{equation}\label{traction_gamma_pre}
			\left(\left((\Lambda^{2h}J^{2h}_{\Gamma})\otimes{\bf I}\right)^{-1}\wt{\mathcal{A}}_3^{2h}{\bf c}^{\star,n+1}\right)_{\bf i}
			= {\mathcal{R}}_{\bf i}\Big(\left((\Lambda^hJ_{\Gamma}^h)\otimes{\bf I}\right)^{-1}(\mathcal{A}_3^h{\bf f}^{\star,n+1}-h_3\omega_1{\bm \eta}^{\star,n+1})\Big), {\bf i}\in I_{\Gamma^c}.
			\end{equation}
		}
		\item{Evaluate the acceleration at all grid points 
			\begin{equation*}
			\underline{\wt{\bf c}}^n= \frac{\wt{\bf c}^{*,n+1}-2\wt{\bf c}^{n}+\wt{\bf c}^{n-1}}{\Delta^2_t},\ \ \ \
			\underline{{\bf f}}^{n} = \frac{{\bf f}^{*,n+1}-2{\bf f}^{n}+{\bf f}^{n-1}}{\Delta^2_t}.
			\end{equation*}
		}
		\item{Compute the corrector at the interior grid points
			\[{\bf c}^{n+1}_{\bf i} = {\bf c}^{*,n+1}_{\bf i} + \frac{\Delta_t^4}{12}\left((\rho^{2h}\otimes{\bf I})(J^{2h}\otimes{\bf I})\right)^{-1}\wt{\mathcal{L}}^{2h}_{\bf i}\underline{{\bf{c}}}^{n},\quad {\bf i}\in I_{\Omega^c},\]
			\[{\bf f}^{n+1}_{\bf i} = {\bf f}^{*,n+1}_{\bf i} + \frac{\Delta_t^4}{12}\left((\rho^{h}\otimes{\bf I})(J^h\otimes{\bf I})\right)^{-1}\hat{\mathcal{L}}^{h}_{\bf i}\underline{{\bf{f}}}^{n},\quad {\bf i}\in I_\Omega^f.\]
		}
		\item{At the interface $\Gamma$, the values ${\bf f}^{n+1}_{\bf i}$  are computed by the continuity of solution
			\begin{equation*}
			{\bf f}^{n+1}_{\bf i} = {\mathcal{P}}_{\bf i}({\bf c}^{n+1}), \quad {\bf i}\in I_{\Gamma^f}.
			\end{equation*}
		}
		\item{At the interface $\Gamma$, the ghost point values in $\wt{\bf c}^{n+1}$ are computed by solving the equation for the continuity of traction
			\begin{equation}\label{traction_gamma_corr}
			%({\Lambda}^{c}_{{\bf i}'}{J}_{{\bf i}'}^{c})^{-1}(\wt{\mathcal{A}}_3^{2h}{\bf c}^{n+1}_{{\bf i}'})
			%= {\mathcal{R}}\left(({\Lambda}^{f}_{\bf i}{ J}^f_{\bf i})^{-1}((\mathcal{A}_3^h{\bf f}^{n+1}_{\bf i})-h_3\omega_1{\bm \eta}^{n+1}_{\bf i})\right), \quad {{\bf i}'}\in I_{\Gamma^c}, \quad {\bf i}\in I_{\Gamma^f}.
			\left(\left((\Lambda^{2h}J^{2h}_{\Gamma})\otimes{\bf I}\right)^{-1}\wt{\mathcal{A}}_3^{2h}{\bf c}^{n+1}\right)_{\bf i}
			= {\mathcal{R}}_{\bf i}\Big(\left((\Lambda^hJ^h_{\Gamma})\otimes{\bf I}\right)^{-1}(\mathcal{A}_3^h{\bf f}^{n+1}-h_3\omega_1{\bm \eta}^{n+1})\Big), \quad {\bf i}\in I_{\Gamma^c}.
			\end{equation}
		}
	\end{itemize}
\end{breakablealgorithm}
~\\

In the Algorithm \ref{first_alg}, we need to solve the linear systems for the continuity of traction at the interface $\Gamma$ in both predictor step (\ref{traction_gamma_pre}) and corrector step (
\ref{traction_gamma_corr}). The linear system matrices of (\ref{traction_gamma_pre}) and (\ref{traction_gamma_corr}) are the same. Therefore, we only present how to solve (\ref{traction_gamma_pre}) in the predictor step.

There are $3n_1^{2h}n_2^{2h}$ unknowns and $3n_1^{2h}n_2^{2h}$ linear equations in (\ref{traction_gamma_pre}). For large problems in three dimensions, it is very memory inefficient to calculate the LU-factorization. Therefore, we use iterative methods to solve the linear system in (\ref{traction_gamma_pre}). In particular, we consider three different iterative methods: the block Jacobi iterative method, the conjugate gradient (CG) iterative method and the preconditioned conjugate gradient iterative method. The detailed methods and a comparison are given in Section \ref{iterative_section}.