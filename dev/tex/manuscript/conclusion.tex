%!TEX root = SISC_elastic_3d.tex
\section{Conclusion}
We have developed a fourth order accurate finite difference method for the three dimensional elastic wave equations in heterogeneous media. To take into account discontinuous material properties, we partition the domain into subdomains such that interfaces are aligned with material discontinuities such that the material property is smooth in each subdomain. Adjacent subdomains are coupled through physical interface conditions: continuity of displacements and continuity of traction.

In a realistic setting, these subdomains have curved faces. We use a coordinate transformation and discretize the governing equations on curvilinear meshes. In addition, we allow nonconforming mesh refinement interfaces such that the mesh sizes in each block need not to be the same. With this important feature, we can choose the mesh sizes according to the velocity structure of the material and keep the grid points per wavelength almost the same in the entire domain. 

The finite difference discretizations satisfy a summation-by-parts property. At the interfaces, physical interface conditions are imposed by using ghost points and mesh refinement interfaces with hanging nodes are treated numerically by the fourth order interpolation operators. Together with a fourth order accurate predictor-corrector time stepping method, the fully discrete equation is energy conserving. We have conducted numerical experiments to verify the energy conserving property, and the fourth order convergence rate. Furthermore, our numerical experiments indicate that there are very little artificial reflections at the interface.

To obtain values of the ghost points, a system of linear equations must be solved. In our formulation, we only use ghost points from the coarse domain, which is more efficient than the traditional approach of using ghost points from both domains.  For large-scale simulations in three dimensions, the LU factorization cannot be used due to memory limitations. We have studied and compared three iterative methods for solving the linear system.

We are currently incorporating the curvilinear mesh refinement algorithm into the open source code SW4 \cite{SW4}. This will enable the algorithm to be used for solving realistic seismic wave propagation problems on large parallel, distributed memory, machines.